
\abstract{
\pagestyle{empty}
\small{
In this thesis, we present several schemes to introduce visual information in motion generation techniques for humanoid robots. We can roughly divide in three levels the motion generation process, from higuest to lowest: the planning level, the pattern generator level and the whole-body motion level. For the planning level we present an approach to inject the motion primitives to the pattern generator through a vector field as a reference velocity. In the pattern generator level we make use of visual servoing with model predictive control (MPC). Since visual servoing with MPC is a nonlinear optimization problem, we propose a linearization scheme in order to keep it as a Quadratic Program (QP) and introduce it within the pattern generator. Finally in the whole-body motion level, we present a 3D reconstruction system to be used in a compliant walking scheme to walk on rough terrain.
}
}
