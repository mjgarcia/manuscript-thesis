
\chapter{Locomotion} 
\label{Chap:Locomotion-Control}

In this chapter we introduce the techniques most commonly used in biped walking generation, the Linearized Inverted Pendulum Model and Kajita's Preview Control. These techniques are widely used and they have become a standard in humanoid walking control. In the next chapters, we will work upon these techniques. The reader is kindly encouraged to read the original papers to have a deeper understanding of these models.


\section{Models of Biped Walking and Kajita's Preview Control}
Most of the works in biped walking are based on the simplified model proposed by Kajita et al. in 1992 \citep{Kajita1992}. This model simplifies the multi-body nature of the robot considering it as a single mass moving as an inverted pendulum. This mass is moving in the $x,y$ plane. It also constraints the trajectory of the CoM to a horizontal plane, as depicted in Fig. \ref{Fig:Inverted-Pendulum}.

\begin{figure}
  \centering
      {\def\svgwidth{1.0\columnwidth}
        \subimport*{Chap3-Locomotion-Control/}
                   {inverted_pendulum.pdf_tex}}
      \caption[]{Humanoid walking modeled as an inverted pendulum. The center of mass is constrained to an horizontal plane. The ZMP corresponds to the point in the sole in which the momentum generated by the inertia and gravity is countered with the momentum generated by the contact forces of the foot with the ground.}
      \label{Fig:Inverted-Pendulum}
\end{figure}

The dynamics of this model is given by,

\begin{eqnarray}
\label{Eq:LIPM_dynamics}
\tau^a_x &=& mgc_y - mc_z \ddot{c}_y, \\
\tau^a_y &=& mgc_x - mc_z \ddot{c}_x
\end{eqnarray}

where $(c_x, c_y, c_z)$ is the 3D position of the CoM, $m$ its mass, and $\tau_x, \tau_y$ the torques produced by the inertia and gravity ($g$) in the base of the pendulum. We can note that the equation in the $y$ axis is completely equivalent to the $x$ axis.

In the contact of the sole with the ground, there exists the countering force $F^p$. The $x,y$ components of this force correspond to friction and avoid sliding. The $z$ component supports the weight of the pendulum and is equal to $F^p_z = mg$, as depicted in Fig.~\ref{Fig:Foot-Sole}.

\begin{figure}
  \centering
      {\def\svgwidth{0.5\columnwidth}
        \subimport*{Chap3-Locomotion-Control/}
                   {foot.pdf_tex}}
      \caption[]{The resultant forces and torques of the robot in the foot are $F^a$ and $\tau^a$ respectively. It corresponds the reactions in the contact with the floor. The resultant of those reactions is $F^p$ and its corresponding torque $\tau^p$. The $z$ component of this resultant $F^p_z$ corresponds to the vertical reaction of the floor. The $x$ and $y$ components correspond to friction forces and avoid sliding.}
      \label{Fig:Foot-Sole}
\end{figure}

The total torques in the base of the pendulum are given by,

\begin{eqnarray*}
\label{Eq:LIPM_total_torques}
\tau^{total}_{x} &=& \tau^p_x + \tau^a_x = F^p_z z^p_x + \tau^a_x, \\
\tau^{total}_{y} &=& \tau^p_y + \tau^a_y = F^p_z z^p_y + \tau^a_y
\end{eqnarray*}

where $(z^p_x, z^p_y)$ is the acting point of $F^p$.

The locomotion is dynamically balanced if the contact forces of the feet with the ground counter the forces due to the inertia and gravity. The Zero Moment Point (ZMP) $(z_x,z_y)$ corresponds to the point in the ground where the total moment in the base of the pendulum is equal to zero \citep{Vukobratovic1972,Vukobratovic2004},

\begin{eqnarray}
0 &=& m g z_x + \tau^a_x, \\
0 &=& m g z_y + \tau^a_y
\end{eqnarray}

substituting the former equations in equation \ref{Eq:LIPM_dynamics}, we derive the ZMP equations,

\begin{equation}
\label{Eq:LIPM_ZMP}
Z = 
\left[
\begin{matrix}
z_x \\ z_y
\end{matrix}
\right] = 
\left[
\begin{matrix}
c_x - \frac{c_z}{g}\ddot{c}_x \\ c_y - \frac{c_z}{g}\ddot{c}_y
\end{matrix}
\right]
\end{equation}

This model is equivalent to the cart-table model. See Fig. ~\ref{Fig:Inverted-Pendulum-Cart-Table}. This model simulates the dynamic of the robot as a running cart above a table. The cart is in a position where it is not statically stable, however, dynamically, if the cart accelerates in an appropriate way, inertia forces will keep it in balance.

\begin{figure}
  \centering
      {\def\svgwidth{0.5\columnwidth}
        \subimport*{Chap3-Locomotion-Control/}
                   {robot-cart-table1.pdf_tex}}
      \caption[]{Equivalent model for humanoid walking. Statically the table is not balanced, dynamically inertia forces balance the table.}
      \label{Fig:Inverted-Pendulum-Cart-Table}
\end{figure}

We take the control variables as the time derivative of the horizontal accelerations of the CoM,

$$
 u_x \stackrel{\mbox{\tiny def}}{=} \dfrac{d}{dt}\ddot{c}_x = \dddot{c}_x.
$$

From the  work of \citep{Kajita2003}, if we suppose that the trajectory has periodic piece-wise constant jerks on a time interval $T$, for discrete time $k$, we can express the CoM dynamics in the $x$-axis as,

$$
c_x(k+1) = c_x(k) + \dot{c}_x(k) T + \ddot{c}_x(k) T^2/2 + \dddot{c}_x(k) T^3/6.
$$

Let us define

$$
\hat{c}_x(k) \equiv 
\left[
\begin{matrix}
c_x(k) \\ 
\dot{c}_x(k)\\
\ddot{c}_x(k) 
\end{matrix}
\right]
$$ 

the state of the robot, defined by its position, velocity and acceleration at time $k$. 

We can express the state of the robot at time $k+1$ in terms of the state $k$ plus the control variable,

\begin{equation}
\label{Eq:state_dynamics}
\left(
\begin{matrix}
c_x(k+1) \\ 
\dot{c}_x(k+1)\\
\ddot{c}_x(k+1) 
\end{matrix}
\right) =
\left(
\begin{matrix}
1 & T & T^2/2 \\
0 & 1 & T \\
0 & 0 & 1
\end{matrix}
\right)
\left(
\begin{matrix}
c_x(k) \\ 
\dot{c}_x(k)\\
\ddot{c}_x(k) 
\end{matrix}
\right) +
\left(
\begin{matrix}
T^3/6 \\
T^2/2 \\
T
\end{matrix}
\right)
u_x(k).
\end{equation}

Finally, using Eq.~\ref{Eq:LIPM_ZMP} and Eq.~\ref{Eq:state_dynamics} we have the basic equations of biped locomotion,

\begin{equation}
\label{Eq:basic_dynamic_equations}
\left\{
\begin{array}{ccc}
 \hat{c}_x(k+1) &=&  A \hat{c}_x(k) + B u_x(k)\\
 z_x(k) &=& C \hat{c}_x(k)
\end{array}
\right.,
\end{equation}

with,

\begin{equation*}
 A = \left(
\begin{matrix}
1 & T & T^2/2 \\
0 & 1 & T \\
0 & 0 & 1
\end{matrix}
\right) \text{, }
{ B} = \left(
\begin{matrix}
T^3/6 \\
T^2/2 \\
T
\end{matrix}
\right) \text{ and }
{ C} = \left(
\begin{matrix}
1 \; 0 \; \dfrac{c_z}{g} \\
\end{matrix}
\right).
\end{equation*}

To solve this control system efficiently, Kajita et al. also  proposed to use Model Predictive Control (MPC). MPC takes into account future information by previewing the behavior of the system for a given horizon. Then, an optimization problem is formulated using a performance index,

\begin{equation}
\label{Eq:Performance-Index}
\min_{U(k)} \sum\limits_{i=k}^{k+N - 1} \frac{1}{2} Q (z_x(i+1) - z_x^{ref}(i+1))^2 + \frac{1}{2}R\dddot{c}_x^2(i),
\end{equation}

where $U(k) = [u(k)\hdots u(k+N-1)]$ is the sequence of the next $N$ controls, $z_x^{ref}(i)$ is a reference ZMP given by a previous footstep planning for instance, $Q$ and $R$ are constants and $N$ is the size of the horizon. The first term of the optimization problem contributes to a minimization of the squared error of the ZMP and the reference ZMP. The second term minimizes the jerks, so we can have smooth trajectories of the CoM.

The optimal controller is given by,

\begin{equation}
\dddot{c}_x(k) = -K_1 \sum\limits_{i=0}^{k} e(i)-K_2c_x(k) - \sum\limits_{j=1}^{N} K_p(j)z_x^{ref}(k+j)
\end{equation}

with $e(i) = z_x(i) - z_x^{ref}(i)$ being the ZMP error w.r.t. the reference ZMP, and $K_1$, $K_2$ and $K_p(j)$ are gains.

Kajita et al. also proposed a second step to correct the effects of the inverted pendulum simplification. This second step takes into account the multi-body dynamics and is re-injected to have better CoM and ZMP trajectories.

\section{Quadratic Programming and Automatic footstep placement}

Wieber in \citep{WieberHumanoids2006} proposed to reformulate the preview control problem as a Quadratic Programming Problem. By applying recursively the dynamics of Eqs.~\ref{Eq:basic_dynamic_equations}, we can express the position, velocity and acceleration  of the CoM in terms of the initial state $\hat{c}_x(k)$ and the sequence of jerks $\dddot{ C}_x(k) \stackrel{\mbox{\tiny def}}{=} \left[ \dddot{c}_x(k), \dddot{c}_x(k+1),...,\dddot{c}_x(k+N-1) \right]^{\transpose}$,

\begin{equation}
 \label{Eq:PosCMHorizon}
 {C}_x(k+1) \stackrel{\mbox{\tiny def}}{=}  \left(
 \begin{matrix}
  c_x(k+1) \\
  \vdots \\
  c_x(k+N-1)
 \end{matrix}
 \right) = {S}_p \hat{c}_x(k) + { U}_p \dddot{C}_x(k),
\end{equation}

\begin{equation}
 \label{Eq:VelCMHorizon}
 \dot{C}_x(k+1) \stackrel{\mbox{\tiny def}}{=}  \left(
 \begin{matrix}
  \dot{c}_x(k+1) \\
  \vdots \\
  \dot{c}_x(k+N-1)
 \end{matrix}
 \right) = {S}_v \hat{c}_x(k) + { U}_v \dddot{ C}_x(k),
\end{equation}

similar expressions can be obtained for the $y$ component. We can also express the ZMP trajectory,

\begin{equation}
 \label{Eq:PosZMPHorizon}
 { Z}_x(k+1) \stackrel{\mbox{\tiny def}}{=}  \left(
 \begin{matrix}
  {z}_x(k+1) \\
  \vdots \\
  {z}_x(k+N-1)
 \end{matrix}
 \right) = { S}_z \hat{c}_x(k) + { U}_z \dddot{ C}_x(k).
\end{equation} 

With the matrices $S_p, S_v, S_z \in \mathbb{R}^{N\times3}$ and $U_p, U_v , U_z \in \mathbb{R}^{N \times N}$ defined as,

\begin{equation*}
  S_{p}=\begin{bmatrix} 1 & T & T^2/2 \\ \vdots & \vdots & \vdots \\ 1 & NT & N^2T^2 \end{bmatrix}, \qquad
  U_{p}=\begin{bmatrix} T^3/6 & 0 & 0 \\ \vdots & \ddots & 0 \\ (1+3N+3N^2)T^3/6 & \cdots & T^3/6 \end{bmatrix},
\end{equation*}

\begin{equation*}
  S_{v}=\begin{bmatrix} 0 & 1 & T \\ \vdots & \vdots & \vdots \\ 0 & 1 & NT \end{bmatrix}, \qquad
  U_{v}=\begin{bmatrix} T^2/2 & 0 & 0 \\ \vdots & \ddots & 0 \\ (1+2N)T^2/2 & \cdots & T^2/2 \end{bmatrix},
\end{equation*}

\begin{equation*}
  S_{z} = \begin{bmatrix} 1 & T & \frac{T^2}2-\frac{z^{c}}{g} \\ \vdots & \vdots & \vdots \\
                          1 & NT & \frac{N^2T^2}{2}-\frac{z^{c}}{g} \end{bmatrix},
\end{equation*}

\begin{equation*}
  U_{z} = \begin{bmatrix}
            \frac{T^3}{6}-\frac{Tz^{c}}{g} & 0 & 0 \\
          \vdots & \ddots & \vdots \\
          [1+3(N-1)+3(N-1)^2]\frac{T^3}{6}-\frac{Tz^{c}}{g} & \cdots & \frac{T^3}{6}-\frac{Tz^{c}}{g}
          \end{bmatrix}.
\end{equation*}

This allows us to rewrite the optimization problem as,

\begin{eqnarray}
\nonumber
\underset{U(k)}{\min} ~~ \dfrac{1}{2} R \| \dddot{{C}}(k) \|^2 + \dfrac{1}{2} Q \|{Z}_x(k+1) - {Z}^{ref}_x(k+1) \|^2,
\label{Eq:MinJerk}
\end{eqnarray}
 
with the same interpretations as in Eq.~\ref{Eq:Performance-Index}.

This optimization problem has the analytical solution,

\begin{equation}
\dddot{{C}}_x(k) = -({U}_z^{\top} {U}_z + \dfrac{R}{Q} {I}_{N \times N})^{-1} {U}_z^{\top}({S}_z \hat{c}_x(k) - {Z}_x^{ref}(k+1)).
\end{equation}

Wieber showed that with this proposal the system is able to reject strong perturbations.

As an evolution of this work, where the footsteps positions (and, correspondingly, the ZMP reference) are fed to the pattern generation, the work of~\citep{HerdtAR2010} introduced automatic footstep placement, i.e. managed the footsteps as free variables in the optimization problem and not as inputs. This reduced the necessary input to a simple stack of reference velocities $(\dot{C}_{x}^{ref}(k+1),\dot{C}_{y}^{ref}(k+1))$. This leads to the optimization problem,

\begin{eqnarray}
\nonumber
 \underset{U(k)}{\min} \; && \dfrac{\alpha}{2} \left\| \dddot{C}_x(k) \right\|^2 + \dfrac{\alpha}{2} \left\| \dddot{C}_y(k) \right\|^2 \\
&& + \dfrac{\beta}{2} \left\| \dot{C}_{x}(k+1) - \dot{C}_{x}^{ref}(k+1) \right\|^2 + \dfrac{\beta}{2} \left\| \dot{C}_{y}(k+1) - \dot{C}_{y}^{ref}(k+1) \right\|^2  \nonumber \\
&& + \dfrac{\gamma}{2} \left\| Z_x(k+1) - Z_x^{ref}(k+1) \right\|^2 + \dfrac{\gamma}{2} \left\| Z_y(k+1) - Z_y^{ref}(k+1) \right\|^2,
\label{Eq:MinJerk}
\end{eqnarray}

with $\alpha$, $\beta$, $\gamma$ being constants that indicate the weight of each term in the optimization problem. The first two terms correspond to the minimization of the jerk, the next two terms correspond to a tracking of a reference velocity and final ones correspond to a tracking of a reference ZMP.

The reference ZMP is defined as,

\begin{eqnarray}
\label{Eq:ZMPReference}
  Z_x^{ref}(k+1) & = V_c \hat{F}_x(k) + V F_x(k) \nonumber \\
  Z_y^{ref}(k+1) & = V_c \hat{F}_y(k) + V F_y(k),
\end{eqnarray}

with $\hat{F}_x(k)$, $\hat{F}_y(k)$ being the current position of the foot on
the ground.

The variables to optimize are

$$
U(k) \stackrel{\mbox{\tiny def}}{=} 
\left(
\begin{matrix}
\dddot{C}_{x}(k) \\
F_x(k) \\
\dddot{C}_{y}(k) \\
F_y(k)
\end{matrix}
\right),
$$

in which $F_x(k)$ and $F_y(k)$ are the next footstep positions in the horizon. In Eq.~\ref{Eq:ZMPReference} we set the reference ZMP to the middle of the support foot. This way, the reference ZMP is not fixed in advanced but is permanently recomputed from the feet position decided by the algorithm. Finally

\begin{equation*}
   V_c = \begin{bmatrix}1 \\ \vdots \\ 1 \\ 0 \\ \vdots \\ 0 \\ 0 \\ \vdots \\ 0  \end{bmatrix} \qquad
   V = \begin{bmatrix} 0 & 0 & 0 & \cdots & 0 \\ \vdots & \vdots
     & \vdots & \ddots\\ 
                                           0 & 0 & 0 & \cdots & 0\\
                           1 & 0 & 0 & \cdots & 0 \\ \vdots & \vdots &
                           \vdots &
                           \ddots \\ 1 & 0 & 0 &\cdots &0\\
                           0 & 1 & 0 & \cdots & 0\\ \vdots & \vdots &
                           \vdots & \ddots \\ 0 & 1 & 0 & \cdots &0\end{bmatrix},
\end{equation*}

are selection matrices that indicate which sampling time falls in which step.

As the terms in the optimization problem defined in Eqs.~\ref{Eq:PosCMHorizon}-\ref{Eq:PosZMPHorizon} and \ref{Eq:ZMPReference} are linear in the variables to optimize, the problem can be written as a Quadratic Program (QP),

\begin{equation}
 \underset{U(k)}{\min} \; \dfrac{1}{2} U(k)^{\transpose} Q(k) U(k) + p(k)^{\transpose} U(k),
\label{Eq:QP}
\end{equation}

under linear constraints arising, among others, from the inclusion of the reference ZMP inside the support polygon~\citep{HerdtAR2010}.

For instance, during the single support phase (in which we have a support foot and a flying one), the constraint ensuring that the ZMP remains inside the support polygon is expressed as:
\begin{equation}
  \begin{bmatrix} d_x(\theta) & d_y(\theta) \end{bmatrix}
  \begin{bmatrix} z_x - f_x \\ z_y - f_y \end{bmatrix} \leq b(\theta)
\end{equation}
where $(f_x,f_y)$ is the foot position, $\theta$ is its orientation,
$d_x(\theta)$, $d_y(\theta)$ are column vectors containing the $x$,
$y$ coordinates of the normal vectors to the feet edges, and $b(\theta)$  is the column vector containing their position with a security margin. For the double support phase, Herdt et al. chose to satisfy the constraint of the reference ZMP at the sampling time kT, and given that the double support phase is chosen to be T long (0.1s for the double support and 0.7 for the single support), no samplings fall strictly in the double support, so they just consider single support constraint in the reference ZMP. This assumption appears to be reasonable enough to generate stable motions as they showed in their experiments.